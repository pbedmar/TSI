\documentclass[12pt]{article} % A4 paper

\usepackage[T1]{fontenc} % Use 8-bit encoding that has 256 glyphs
\usepackage[utf8]{inputenc}

\usepackage[spanish, es-tabla]{babel} % Selecciona el español para palabras introducidas automáticamente, p.ej. "septiembre" en la fecha y especifica que se use la palabra Tabla en vez de Cuadro
\usepackage{graphics,graphicx, float} %para incluir imágenes y colocarlas
\usepackage{booktabs}

\parskip=3pt

\usepackage[
    a4paper,
    left=2.8cm,
    right=2.7cm,
    top=2.5cm,
    bottom=2.5cm
]{geometry}

\par

%----------------------------------------------------------------------------------------
%	TÍTULO Y DATOS DEL ALUMNO
%----------------------------------------------------------------------------------------

\title{	

\vspace{-2.5cm}
\LARGE \textbf{Técnicas de los Sistemas Inteligentes} \\
\LARGE Práctica 2: Satisfacción de restricciones con MiniZinc \\[0.5em]
\large Curso 2021-2022 \par
\large Pedro Bedmar López - 75935296Z \\
\normalsize pedrobedmar@correo.ugr.es \par
\large Grado en Ingeniería Informática
\vspace{-7pt}
\rule{\textwidth}{0.4pt}
\vspace{-2cm}
}

\date{}

%----------------------------------------------------------------------------------------
% DOCUMENTO
%----------------------------------------------------------------------------------------

\begin{document}

\clearpage
\maketitle % Muestra el Título

\section{Problema de las monedas}
En este ejercicio se pretende resolver el problema de que dada una cantidad de dinero, se devuelve en monedas de 1, 2, 5, 10, 20 y 50 céntimos y en monedas de 1 y 2 euros.

\subsection{Apartado a)}
En este apartado calculamos cualquier solución válida, o sea, cualquier combinación de esas monedas que sume exactamente el importe introducido. Al no existir ninguna restricción aparte del uso de esas monedas específicas, el número de soluciones es muy alto y crece exponencialmente (como también lo hace el tiempo de ejecución).
% TODO: Crece exponencialmente el número de soluciones y el tiempo?

En la siguiente tabla se muestran los resultados de ejecución en 4 situaciones. La solución encontrada se expresa como un vector [c1,c2,c5,c10,c20,c50,e1,e2], donde cx representa la cantidad de monedas de x céntimos y ex la cantidad de monedas de x euros.

\begin{table}[H]
\centering
\begin{tabular}{@{}c|lll@{}}
\toprule
Importe &
    \multicolumn{1}{c}{\begin{tabular}[c]{@{}c@{}}Primera solución encontrada\\ y número de monedas de la misma\end{tabular}} &
    \multicolumn{1}{c}{\begin{tabular}[c]{@{}c@{}}Número total\\ de soluciones\end{tabular}} &
    \multicolumn{1}{c}{\begin{tabular}[c]{@{}c@{}}Runtime\\ (en segundos)\end{tabular}} \\ \midrule
0.17€ & {[}17,0,0,0,0,0,0,0{]} $-> 17$ monedas   & 28     & 0.286  \\
1.43€ & {[}143,0,0,0,0,0,0,0{]} $-> 143$ monedas & 17952  & 2.692  \\
2.35€ & {[}235,0,0,0,0,0,0,0{]} $-> 235$ monedas & 150824 & 25.764 \\
4.99€ & {[}499,0,0,0,0,0,0,0{]} $-> 499$ monedas & 6224452  & 2002 \\ \bottomrule
\end{tabular}
\caption{Resultados del apartado a) del problema de las monedas.}
\label{tab:my-table}
\end{table}

\subsection{Apartado b)}
En este caso, añadimos restricciones extra que impiden que con monedas de céntimo podamos representar un euro o más. Debido a esto el espacio de búsqueda se va a reducir y el tiempo de ejecución por tanto también disminuye. 

\begin{table}[H]
\centering
\begin{tabular}{@{}c|lll@{}}
\toprule
Importe &
    \multicolumn{1}{c}{\begin{tabular}[c]{@{}c@{}}Primera solución encontrada\\ y número de monedas de la misma\end{tabular}} &
    \multicolumn{1}{c}{\begin{tabular}[c]{@{}c@{}}Número total\\ de soluciones\end{tabular}} &
    \multicolumn{1}{c}{\begin{tabular}[c]{@{}c@{}}Runtime\\ (en segundos)\end{tabular}} \\ \midrule
0.17€ & {[}17,0,0,0,0,0,0,0{]} -\textgreater 17 monedas  & 28    & 0.342 \\
1.43€ & {[}43,0,0,0,0,0,1,0{]} -\textgreater 143 monedas & 284   & 0.311 \\
2.35€ & {[}35,0,0,0,0,0,2,0{]} -\textgreater 235 monedas & 324   & 0.412 \\
4.99€ & {[}99,0,0,0,0,0,4,0{]} -\textgreater 499 monedas & 13098 & 2.200 \\ \bottomrule
\end{tabular}
\caption{Resultados del apartado b) del problema de las monedas}
\label{tab:my-table}
\end{table}

\subsection{Apartado c)}
En este caso observamos que los tiempos de ejecución son menores o iguales que en el caso anterior.

% Please add the following required packages to your document preamble:
% \usepackage{booktabs}
\begin{table}[H]
    \centering
    \begin{tabular}{@{}c|ll@{}}
    \toprule
    Importe &
      \multicolumn{1}{c}{\begin{tabular}[c]{@{}c@{}}Primera solución encontrada\\ y número de monedas de la misma\end{tabular}} &
      \multicolumn{1}{c}{\begin{tabular}[c]{@{}c@{}}Runtime\\ (en segundos)\end{tabular}} \\ \midrule
    0.17€ & {[}0,1,1,1,0,0,0,0{]} -\textgreater 3 monedas & 0.283 \\
    1.43€ & {[}1,1,0,0,2,0,1,0{]} -\textgreater 5 monedas & 0.327 \\
    2.35€ & {[}0,0,1,1,1,0,0,1{]} -\textgreater 4 monedas & 0.338 \\
    4.99€ & {[}0,2,1,0,2,1,0,2{]} -\textgreater 8 monedas & 0.376 \\ \bottomrule
    \end{tabular}
    \caption{Resultados del apartado c) del problema de las monedas}
    \label{tab:my-table}
    \end{table}

\subsection{Apartado d)}
\subsubsection*{¿Qué ocurriría si, usando la codificación (a) para encontrar todas las soluciones, el importe buscado es mucho mayor?} 
Ocurre que el espacio de búsqueda crece exponencialmente, y de la misma forma lo hace el tiempo de ejecución, por lo tanto llega un punto en el que el problema es inabarcable. 

\subsubsection*{¿Se podría encontrar alguna solución (usando la codificación de (a) o cualquier otra) de este problema con un importe del orden de los millones de euros?}
Sí, una codificación donde únicamente existan monedas de un céntimo sería más eficiente y podría resolver el problema con tiempos de ejecución menores.


\section{Problema de los horarios}


\section{Problema lógico}


\section{Problema de la asignación de tareas}


\section{Problema del coloreado de grafos}


\end{document}